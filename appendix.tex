\chapter{数式の記述}

本付録では,数式の記述例を示す.
本手引のクラスファイル \texttt{kzthesis.cls} 独自のコマンドの使用法の説明も兼ねているので,
参考にすること.


\section{記述例}

まず,受動性の定義を与えておく \Cite{03}.
システムの入力 $\bm{u}$ と出力 $\bm{y}$ が同じ次元であるとする.
このとき,ある有限な正の定数 $\gamma_0{}^2$ に対して次式が成り立つならば,
システムは受動的であるという.
\begin{equation}
	\int_0^t \bm{y}^T(\tau) \bm{u}(\tau) d\tau \ge -\gamma_0{}^2, \qquad {}^\forall t>0				\label{eq:passivity}
\end{equation}

一方,$n$ 自由度ロボットの動力学は一般に次のような微分方程式に従う.
\begin{equation}
	\bm{u} = \bm{M}(\bm{q}) \ddot{\bm{q}}
		 + \frac{1}{2} \dot{\bm{M}}(\bm{q}) \dot{\bm{q}}
		 + \bm{S}(\bm{q},\dot{\bm{q}}) \dot{\bm{q}}
		 + \bm{g}(\bm{q})
		 + \bm{d}(\dot{\bm{q}})																		\label{eq:em}
\end{equation}
ただし,
$\bm{q} \in \Re^n$ は関節変位ベクトル,
$\bm{u} \in \Re^n$ は関節駆動力ベクトルである.
右辺第一項は慣性項であり,$\bm{M}(\bm{q}) \in \Re^{n \times n}$ は実対称正定な慣性行列である.
第二,三項は非線形項であり,$\bm{S}(\bm{q},\dot{\bm{q}}) \in \Re^{n \times n}$ は歪対称行列となる.
$\bm{g}(\bm{q}) \in \Re^n$ はポテンシャル項である.
$\bm{d}(\dot{\bm{q}}) \in \Re^n$ は摩擦等による散逸項であり,
その各成分は $\dot{\bm{q}}$ の対応する成分と常に同符号である.
$K(\bm{q},\dot{\bm{q}})$,$P(\bm{q})$ をそれぞれ運動エネルギー,ポテンシャルエネルギーとすると,
以下のような関係が成り立つ.
\begin{align}
	K(\bm{q},\dot{\bm{q}}) &= (1/2) \, \dot{\bm{q}}{}^T \bm{M}(\bm{q}) \dot{\bm{q}}					\\
	\bm{g}(\bm{q}) &= (\partial P(\bm{q})/\partial \bm{q}{}^T){}^T
\end{align}
このとき,
ロボットの全内部エネルギーは $E(\bm{q},\dot{\bm{q}}) = K(\bm{q},\dot{\bm{q}}) + P(\bm{q})$ となる.

ロボットシステムは,
速度出力 $\bm{y} = \dot{\bm{q}}$ に関して受動的であることが知られている.
すなわち,式 \Eq{em} から,
\begin{align}
	\int_0^t \dot{\bm{q}}{}^T \bm{u} \, d\tau
		&= \int_0^t \dot{\bm{q}}{}^T \left\{
			\bm{M} \ddot{\bm{q}}
			+ \frac{1}{2} \dot{\bm{M}} \dot{\bm{q}}
			+ \bm{S}(\bm{q},\dot{\bm{q}}) \dot{\bm{q}}
			+ \bm{g}(\bm{q})
			+ \bm{d}(\dot{\bm{q}}) \right\} d\tau													\nonumber \\
		&= \int_0^t \frac{d}{d\tau} \left\{ K(\bm{q},\dot{\bm{q}}) + P(\bm{q})) \right\} d\tau
			+ \int_0^t \dot{\bm{q}}{}^T \bm{d}(\dot{\bm{q}}) \, d\tau 								\nonumber \\
		&= E(\bm{q}(t),\dot{\bm{q}}(t)) - E(\bm{q}(0),\dot{\bm{q}}(0))
			+ \int_0^t \dot{\bm{q}}{}^T \bm{d}(\dot{\bm{q}}) \, d\tau 								\nonumber \\
		&\ge - E(\bm{q}(0),\dot{\bm{q}}(0)) = -\gamma_0{}^2											\label{eq:limit}
\end{align}
となり,式 \Eq{passivity} が成立する.
ただし,$\bm{S}$ の歪対称性と $\dot{\bm{q}}{}^T \bm{d}(\dot{\bm{q}}) \ge 0$ を用いた.
上式から,$\gamma_0{}^2$ はロボットの初期内部エネルギーと解釈できる.

\chapter{結論}

\section{本論文のまとめ}
本研究では、比較的安価なRGB-Dカメラを用いた配管アイソメ図の自動生成手法を提案した。
さらに、深層学習を用いた画像認識を導入することで、配管の3次元位置姿勢を推定する6D姿勢推定を実現し、効率的にアイソメ図を生成できることを示した。

本研究では、配管6D姿勢推定において、RGB画像のみを使用する方法とRGB-D画像を使用する方法を比較した。
RGB画像ベースの方法は処理時間が短い一方で、RGB-D画像ベースの方法では深度情報を活用することで、点群データによるポイントマッチングを行い、高い精度を達成した。
そのため、本研究では、深層学習を用いた6D姿勢推定において、RGB-D画像ベースの深層学習モデルを採用した。

さらに、検証実験では大規模な配管設備における有効性を評価するため、仮想環境での実験を実施した。
ただし、仮想環境で取得されるDepth画像には実環境に比べてノイズが存在しないため、センサモデルを導入して実環境との差異を補正した。
得られたDepth画像を用いてアイソメ図生成を行った結果、仮想環境においても比較的大規模な配管設備でアイソメ図を適切に生成できることが確認された。
しかし、Depthノイズレベルが高いセンサを使用した場合、アイソメ図が破綻する問題が発生したため、適切な精度を有するセンサの選定が重要であることが分かった。

\section{今後の課題}
本研究で使用したRGB-D画像は、配管の接続部が必ず画像内に収まる状況を前提にアイソメ図を作成した。
しかし、接続部が画像内に収まらない場合や、画像上で配管の接続部が隠れるオクルージョンの状況下では、アイソメ図の作成が困難であった。
この課題に対処するため、複数視点から撮影した画像を統合してアイソメ図を作成する手法を検討する必要がある。
また、RGB-Dカメラを使用して配管設備全体を点群データとして取得し、点群データを基にしたアイソメ図作成も考えられる。

続いて、作成されたアイソメ図では、接続部のペアが存在しない場合の配管寸法を記載することができなかった。
これは本手法が接続部の認識を主な対象としているため、地面に接続されている直管を認識できなかったためである。
この問題を解決するため、接続部のペアが存在しなかった場合に、直管を認識し設置面との距離を表示できるように改良を行いたい。

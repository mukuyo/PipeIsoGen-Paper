\chapter{%
その他の注意事項}

本章では,論理構造,体裁以外で,
論文作成の際に重要だと思われる注意点について列挙しておく.


\section{論文のオリジナリティ}

学術論文においてはオリジナリティが重視されるため,
「自分の仕事」と「他人の仕事」を明確に分けておくことが不可欠である.
自分がある大きなテーマの一部を担当しているなら,
その大きなテーマの中での自らの寄与を明確にしなければならない.
他人のアイデアや仕事を,さも自分にオリジナリティがあるように書いてはならない.
直接指導を受けている教員のアイデアであっても「(だれそれ)によって提案された(なになに)~」と明示すること.

論文は,オリジナルな結果を発表するものであり,
スタッフや先輩に言われるがままに実験してその結果を報告する,
という内容であれば,
たとえ実験の趣旨を十分理解してデータをとったとしても,
データを取った人の論文として発表すべきものではない.
データ取得にどれほど苦労してもオリジナリティがなければ,
謝辞で「実験・調査・観察を手伝ってくれた人」として載せるのが妥当である.
「自らのオリジナリティが評価される」ということを常に意識し,
自らの頭を使って考えること.

「先生(先輩)の研究を手伝っている」という意識で卒論・修論を書いている学生が散見されるが,
これは大きな間違いである.
修士はもちろん,学士であっても,自ら論文を書く能力を育むことが求められている.
学生諸君が論文作成に主体的かつ積極的に取り組むことが必須である.
先生や先輩の方がむしろ,指導を通してその手伝いをしているのである.


\section{いつ論文を書き始めるか?}

研究成果がすべて出てからでは効率が悪い.
ある程度方針が固まったら,論文を書き始めるべきである.
結果が出ていなくとも,タイトル,目次,緒言,(期待する)結言は書くことができる.
論理構成をまとめることで研究の流れが整理され,それ自体が研究を進めていく上での指針となる.
したがって,
研究室に配属され,研究テーマが決定したら,
早速本手引を参考にして論文を書き始めるべきである.


\section{単位系}

原則として国際単位系すなわちSI単位系を用いること.
例えば,[sec] という記述は古い.[s] とすること.
ただし,グラフの軸等の単位で直感的な分かりやすさを重要視してSI単位でないものを使用することは構わない.
例えば,目標軌道を [deg] で設定したならば,
その軌道制御実験結果のグラフは [rad] ではなく [deg] で書いた方が良い.
また,単位は通常の Roman で記述し,[$deg$] のようにイタリックとしない.
数式としての表現が必要な単位であっても,[$m/s^2$] ではなく [m/s${}^2$] として,変数と区別すること.


\section{一覧性}

論文には,一覧性が求められる.
例えば,比較すべき二つのグラフがある場合,
それらが二ページに渡っていては不便である.
多少グラフが小さくなっても,一覧できるように載せること.


\section{分量}

同じ研究成果を表現するための文章量は少ない方が良い.
「少ないとさみしい」という(?)勘違いな理由でデータを羅列し,
論文の分量を膨らますのはもってのほかである.
記述は常に必要十分を心掛け,無駄な表現,言い回し,データがないかどうかチェックすること.
また,必要な情報をすべて記述しているかをチェックすること.
基準は,その論文を読んだ人が,
論文で主張されている理論,シミュレーション,実験を再現して同じ結果を出せるか,という視点で見ると良い.
再現性のない(再現できない)ものは信用されないと考えること.


\section{カラー}

原則として,論文ではカラーは使わない.
これは,あらゆる印刷での可読性を維持するために重要である.
自分がカラーで見ることができるからと言って,
他の人もカラーで見ることができるとは思わないこと.

グラフでは,安易にカラーを使いがちであるが,
その前に白黒で表現できないか考えること.
たとえカラーを使わざるを得ないと判断したとしても,
白黒で印刷した場合の可読性を最大限確保するように心掛けること.

写真等は,カラーでも白黒印刷して分からなくなる訳ではないので,カラーにしておく方が良い.
その場合も,白黒印刷を意識した撮影を心掛けるべきである.

ただし,最近はカラーでの印刷環境も普及しつつあり,
また PDF ファイルでの閲覧も可能であるため,修論・卒論の場合にはカラーの使用もあり得る.


\section{曖昧な表現}

立命館の学生の文章に非常に多く見られる表現が,「~といえる」「~とわかる」である.
これらはほとんどの場合,意味を曖昧にする効果しか持たず,削除しても全く問題ない.
理系の文章においては,断定すべきところは断定し,不確実なところはどう不確実か明示するべきであり,
このような曖昧な語尾の言い回しはしないこと.


\section{厳密な表現}

前節とは逆に,不用意に厳密な表現も多々見られる.
その最たるものが「最適な~」「~を検証する」である.

何らかの条件が「最適である」ことを示すためには,
\begin{enumerate}
	\item 最適性を数値的に表現できる評価関数を定義する
	\item その条件下において評価関数が最大(最小)であることを証明する
\end{enumerate}
というプロセスが不可欠である.
このプロセスを経ていない条件に対して「最適な~」という言葉を使ってはならない.

また,何らかの仮説が事実であることを「検証した」と主張するためには,
\begin{enumerate}
	\item 仮説の妥当性を理論的に証明する
	\item 上記の証明が正しいことを示すため,および,実装上の問題が生じないことを示すため,
			シミュレーションもしくは実験,できればその両方で仮説の妥当性を確認する.
\end{enumerate}
ことが必要である.
シミュレーションや実験をしただけでは「検証した」と主張してはならない.


\section{情緒的な表現}

理系の文章においては,客観的な記述がすべてであり,主観的感情を含む情緒的表現は一切省く必要がある.
例えば,「非常に高精度な(単に高精度と書けばよい.程度を示したいなら定量的に精度を示す)」,
「~してしまった(~した,でよい)」,
「この実験は非常に難しかった(感想など不要)」,
など,自分の曖昧な主観を基準とした情緒的表現がないか,今一度確認すべきである.


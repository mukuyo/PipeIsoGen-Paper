\chapter{結論}

\section{本論文のまとめ}
本論文ではRGB-Dカメラを用いた深層学習による配管の物体検出のネットワーク開発から始まり,姿勢推定や距離情報算出などのアイソメ図作成までの開発を行った.
特に,配管の物体検出においてはRXDネットワークを提案し,RGB-D画像から既存のネットワークより優れた精度を示すことができた.RGB-DカメラはRGB画像と比較すると
Depth画像を使用できることから,奥行き情報を取得できるだけでなく暗所の環境でも安定した認識結果を出力することができた.しかし,RGB-Dカメラの精度の影響で遠くにあるオブジェクトの検出することが困難であった.
また,姿勢推定においては既存のGen6Dネットワークを用いて姿勢推定を行った.Gen6Dは複数物体の姿勢推定ができなかったため,検出器をRXDネットワークに変更することで複数検知を実装することができた.
最後に,アイソメ図を作成するにおいて距離情報を取得するために,姿勢推定されたデータを元にオブジェクト間の距離を求めることができた.

\section{今後の課題}
本研究ではRGB-D画像からRXDネットワークより曲管および,T字管の検出に成功した.しかし,6D姿勢推定ではRGB画像からColmapによって得られた点群データを使用しているため,今後はDetph画像を用いて3次元モデルの生成を行いたい.
また,6D姿勢推定をDepth画像にも対応したネットワークの提案も実現したい.最後に,アイソメ図作成までのステップまで至らなかったため,得られた情報から図面を描画できるシステムを構築したい.
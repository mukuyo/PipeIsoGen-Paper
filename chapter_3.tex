\chapter{アイソメトリック図生成}

\subsection{概要と生成手法} 
本研究では、配管6D姿勢推定の結果を基に、アイソメトリック図を生成する手法を提案する。
最初に、各配管の接続部の姿勢情報を利用して配管間の接続関係を推定する。
アイソメ図を作成するには、接続部のペアを特定し、正確に配管を接続する必要がある。
これには、6D姿勢推定によって得られた位置と姿勢の情報を用いる。
次に、推定された接続関係に基づいて、配管を線で結び、アイソメ図を生成する。

\subsection{接続関係推定(ペアマッチング)} 
配管間の接続関係を特定するために、接続部の方向ベクトルを計算し、そのベクトル間の角度差が小さい接続部をペアとして認識する。
曲管では出口が2つあるため2つの方向ベクトルを、T字管では3つの方向ベクトルを算出する。
これらのベクトルの角度差が小さく、かつ距離が最も短い接続部をペアとして認識することで、接続関係を推定する。
ただし、地面に接続している配管のようにペアが存在しない場合には、ペアマッチングを適用しない。

\subsection{配管経路の探索} 
接続された配管の経路を効率的に探索するため、深さ優先探索(Depth First Search, DFS)アルゴリズムを使用する。
DFSは、グラフ構造内のノードを一つの経路で可能な限り深く進み、行き止まりに達した際に戻って別の経路を探索するアルゴリズムである。
この手法により、すべての接続部を効率的に訪問し、再帰的に探索を進めることが可能となる。
探索の起点は最も左端に位置する配管とし、接続情報に基づいて配管を順次描画する。

\subsection{アイソメ図の描画} 
アイソメ図の描画には、Pythonのezdxfライブラリを使用する。
ezdxfはDXFファイルの作成に特化しており、直線や円弧などの描画が可能である。
配管の寸法情報は、カメラ座標系で取得した3次元位置データを基に計算される。
配管経路探索で得られた接続情報をもとに、ezdxfを用いて正確なアイソメ図を描画する。
この方法により、配管システムの構造を視覚的に表現する図面を効率的に作成することができる。

% \section{アイソメ図変換方法}
% アイソメ図作成には配管の特徴を活かした効率的な手法を提案する.図2.2に一部配管の例を示す.
% 一般的な配管は両端部分の曲管やT字管などのつなぎ目を除くと直管であるという特徴がある.そのため,両端の曲管がどの方向を向いているのかを推論できれば向かい合っている曲管のペアを見つけられ,その間を直線で結ぶことで
% アイソメ図を描画することができる.そのため,本研究においては配管全体を認識するのではなく,配管のつなぎ目である曲管及びT字管を物体検出と姿勢推定を用いて推論する.

% \begin{figure}[htbt]
% 	\centering
% 	 \includegraphics[height=55mm]{pipe.eps}
% 	 \caption{配管の検出例}
% 	 \label{fig:f2}
% \end{figure}


% \section{データセット収集}
% 深層学習による認識ネットワークにはデータセットの数量が多いほど精度とロバスト性が向上する.
% それは様々な場面での配管の写真を学習することによってどの環境においても対応できる汎用性が高まることを意味している.本研究使用するデータセットの一部を図3.2に示す.
% 配管には曲管やT字管や直管が含まれており,この画像内の中から曲管とT字管を全て認識できることを目標とする.また,Depth画像の有効性を示すためにテスト画像では
% 暗闇の中に配管を設置したデータセットを用意した.Depth画像は光の影響を受けにくいことから,暗闇の中でも配管を認識できるかを検証する.\\
% 収集したデータはラベリング作業を行う.これは深層学習するにおいての正解データとして,予め画像内のどの部分が曲管又はT字管であるかをアノテーションする必要がある.
% 本研究では配管画像に対して曲管,T字管の二つのクラスに分けてラベリング作業を行った.

% 6D姿勢推定のデータセットにはColmapを使用して点群データを取得する.Colmap2D画像から点群を再構築するために使用されるソフトウェアである.
% この2D画像は異なる視点から撮影された同じオブジェクトの画像を複数枚利用することで3次元情報を復元することができる.
% そのため,本研究では曲管とT字管の周囲をそれぞれ撮影し,Colmapを使用することで点群データを取得した.
% 図3.6では姿勢推定を行った後,得られた出力の評価を行う際に使用する.
% しかし,点群データを取得してもColmapで生成されたデータには距離情報が含まれていないため,別途Depth画像を使用してアイソメ図作成の際に使用しなければいけない.

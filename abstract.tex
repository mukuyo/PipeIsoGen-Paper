\section*{摘要}

近年の建築業界ではBIM(Building Information Modeling)と呼ばれるコンピュータ上に現実と同じ建物の立体モデルを再現し,可視化するワークフローが注目されている.
従来の配管BIMでは、高精度なLIDARセンサを用いて配管モデルを推定していたが、LIDARは高価であり、処理効率にも課題があった。

本研究では、LIDARセンサに代わり低コストなRGB-Dカメラを使用することや、深層学習を活用した画像認識技術を用いることで、配管のアイソメトリック図(アイソメ図)を効率的に生成することを目指す。
また、提案手法の有効性を検証するため、仮想環境において大規模な配管システムを構築し、アイソメ図を生成する検証を行った。

\section*{%
摘要}

近年の建築ではBIM(Building Information Modeling)と呼ばれるコンピュータ上に現実と同じ建物の立体モデルを再現し,
可視化するワークフローが注目されている.従来の配管BIMは高精度なLidarセンサを用いて配管モデルの推定を行なわれていたが,
振動に弱く高価である.そのため,Lidarセンサより安価であるRGB-Dカメラを使用し、従来の点群データのみを用いた3D再構築を行わず,
取得画像と関連する点群データに基づき配管のアイソメ図を作成を目標とする。また、物体検出及び姿勢推定には機械学習を使用し、独自のRXDネットワークの提案により高精度かつ高効率化を図る。


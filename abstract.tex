\section*{論文要旨}

近年の建築業界ではBuilding Information Modeling(BIM)と呼ばれるコンピュータ上に現実と同じ建物の立体モデルを再現し,可視化するワークフローが注目されている.
従来の配管BIMでは、高精度なLIDARセンサを用いて配管モデルを推定していた。
しかし、LIDARは高価であり、データ処理に時間を要するという課題があった。

本研究では、LIDARセンサの代替として低コストなRGB-Dカメラを活用し、さらに深層学習を用いた画像認識技術を組み合わせることで、配管のアイソメトリック図(アイソメ図)を効率的に生成する手法を提案する。
深層学習には、3次元位置姿勢を推定する6D姿勢推定を用いることによって、配管の向きや位置を正確に特定し、配管の接続関係を把握した。
また、複数の6D姿勢推定モデルを評価指標に基づいて比較検証し、本提案手法に適したモデルを選定した。
さらに、提案手法の有効性を評価するため、仮想環境において大規模な配管システムを構築し、アイソメ図の生成が可能であることを確認した。


\section*{%
摘要}

近年の建築ではBIM(Building Information Modeling)と呼ばれるコンピュータ上に現実と同じ建物の立体モデルを再現し,
可視化するワークフローが注目されている.従来の配管BIMは高精度なLidarセンサを用いて配管モデルの推定を行なわれていたが,
振動に弱く高価である.そのため,Lidarセンサより安価であるRGB-Dカメラを使用し,従来の点群データのみを用いた3D再構築を行わず,
取得画像と関連する点群データに基づき配管のアイソメ図を作成を目標とする.アイソメ図を作成する段階までには至らなかったが,
機械学習を用いた配管の物体検出及び6D姿勢の推定に成功した.物体検出においてはRGB画像とDepth画像のそれぞれの特性を維持したまま相関を共有する
RXDネットワークを提案し,他のネットワークよりもAP評価指標において最も優れた検出精度を示した.また,配管姿勢推定では既存のGen6DモデルをRXDネットワークによって
複数オブジェクト検出を可能にし,アイソメ図作成に必要な配管の6D姿勢やスケール情報を取得することができた.

